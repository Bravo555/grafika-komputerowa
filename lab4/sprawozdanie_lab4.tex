\documentclass[a4paper]{article}
\usepackage[utf8]{inputenc}
\usepackage{polski}
\usepackage[polish]{babel}
\usepackage[T1]{fontenc}

\title{TEMAT ĆWICZENIA: OpenGL – interakcja z użytkownikiem}
\author{Marcel Guzik}

\begin{document}
\maketitle
\clearpage

\section{Wprowadzenie}
Ćwiczenie ma za zadanie pokazać, jak przy pomocy funkcji biblioteki OpenGL z
biblioteką GLUT można zrealizować prostą interakcję, polegające na sterowaniu
ruchem obiektu i położeniem obserwatora w przestrzeni 3D. Do sterowania służyla
będzie mysz. Ponadto zostaną zilustrowanie sposoby prezentacji obiektów
trójwymiarowych w rzucie perspektywicznym.

\section{Teoria}


\section{Wykonanie programu}

\end{document}