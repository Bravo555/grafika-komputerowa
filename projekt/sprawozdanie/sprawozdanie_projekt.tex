\documentclass[a4paper]{article}
\usepackage[a4paper]{geometry}
\usepackage[utf8]{inputenc}
\usepackage{polski}
\usepackage[polish]{babel}
\usepackage[T1]{fontenc}
\usepackage{graphicx}
\usepackage{verbatim}
\usepackage{float}
\usepackage{minted}
\usepackage{microtype}
\usepackage{shortvrb}
\usepackage{amsmath}

\let\olditemize=\itemize \let\endolditemize=\enditemize \renewenvironment{itemize}{\olditemize \itemsep0em}{\endolditemize}

\title{TEMAT ĆWICZENIA: OpenGL – projekt systemu słonecznego}
\author{Marcel Guzik}

\begin{document}
\section{Wprowadzenie}

Celem ćwiczenia jest wykonanie projektu inkorporującego wszystkie techniki
OpenGL poznane podczas kursu, wykonując interaktywny model systemu słonecznego.

\section{Założenia projektu}

Aplikacja realizująca interaktywny model systemu słonecznego powinna
charakteryzować się następującymi cechami:

\begin{itemize}
    \item Implementować realistyczny system oświetlenia przez oświetlanie i
          cieniowanie Phonga
    \item Pokazywać relatywnie dokładne powierzchnie planet dzięki użyciu
          tekstur (użyć minimalnie tekstury koloru oraz tekstury normalnych)
          Dodatkowo, zapewnić odpowiednie kolory, tj. brać pod uwagę mechanizm
          korekcji gamma używany w przestrzeni kolorów sRGB. Należy zatem
          zwrócić uwagę czy wczytywane tekstury mają zaaplikowaną korekcję
          gamma, i jeśli tak przekształcić je do liniowej przestrzeni kolorów
          dla celów wyliczeń światła. Finalna korekcja gamma zostanie wykonana
          przez aplikację na gotowej do wyświetlenia klatce.
    \item Udostępniać wybór użytkownikowi między realistyczną a poglądową skalą
    \item Zapewniać intuicyjne sterowanie kamerą
    \item Poruszać planetami wg. ich orbit dookoła słońca, które nie są
          dokładnymi okręgami lecz elipsami, którą również należy wyświetlić
          użytkownikowi
    \item Zapewnić kontrolę czasu by przyśpieszać lub spowalniać obieg planet.
\end{itemize}

\section{Duże zmiany}

\subsection{Zmiana używanej biblioteki użytkowej OpenGL z GLUT na GLFW}

Biblioteka GLUT powstała dość dawno temu i w związku z tym posiada pewne
znaczące ograniczenia projektowe. Nie jest już utrzymywana, a jej
otwartoźródłowy port freeglut może dodawać nowe funkcjonalności, jednak musi
zachowywać kompatybilność z oryginalną biblioteką, zatem w praktyce nie jest w
stanie naprawić niektórych błędów GLUT.

W związku z powyższym, zdecydowano się wykorzystać bibliotekę GLFW, która daje
programiście większą kontrolę nad aplikacją, zapewnia więcej funkcjonalności,
przykładowo:

\begin{itemize}
    \item W GLFW programista sam tworzy główną pętlę programu, gdzie w GLUT
          programista musi wywołać funkcję glutMainLoop która nigdy nie zwraca.

    \item GLFW zapewnia dostęp do ``surowych'' danych poruszeń myszą, tj.
          nieprzeskalowanych i niezakcelerowanych zgodnie z ustawieniami
          akceleracji kursora myszy. Jendakże przede wszystkim poruszenia myszką
          wzdłuż obu osi są udostępniane jako liczba zmiennoprzecinkowa, dzięki
          czemu poruszenia kamerą nie będą ``skokowe''.
\end{itemize}


\end{document}
